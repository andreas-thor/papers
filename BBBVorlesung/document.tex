\documentclass{style/lni}
% in englisch stattdessen:
%\documentclass[english]{lni}

\IfFileExists{style/latin1.sty}{\usepackage{style/latin1}}{\usepackage{isolatin1}}

\usepackage{graphicx}
\usepackage{style/fancyhdr}
\usepackage{listings} %if lstlistings is used
\usepackage{style/changepage} %for changing topmargin on first page
\usepackage[figurename=Abb., tablename=Tab., small]{caption}[2008/04/01]
\renewcommand{\lstlistingname}{List.}    % Listingname hei�t nun List. 
 
%Beginn der Seitenz�hlung f�r diesen Beitrag
\setcounter{page}{11}

%Kopfzeileneinstellungen
\pagestyle{fancy}
\fancyhead{} % L�scht alle Kopfzeileneinstellungen
\fancyhead[RO]{\small $<$Vorname Nachname [et. al.]$>$(Hrsg.): $<$Buchtitel$>$, \linebreak Lecture Notes in Informatics (LNI), Gesellschaft f�r Informatik, Bonn $<$Jahr$>$ \hspace{5pt} \thepage \hspace{0.05cm}}
\fancyfoot{} % L�scht alle Fu�zeileneinstellungen
\renewcommand{\headrulewidth}{0.4pt} %Linie unter Kopfzeile 
\setcounter{footnote}{0}

\author{Voxxcxc rname1 Nachname1\footnote{Einrichtung/Universit�t, Abteilung,
Anschrift, Postleitzahl Ort, emailadresse@author1} \, Vorname2 Nachname2\footnote{Einrichtung/Universit�t, Abteilung, Anschrift, Postleitzahl Ort, emailadresse@author2} \ und weitere Autorinnen und Autoren in der gleichen Notation}
\title{Der Titel des Beitrags, der auch etwas l�nger sein und �ber zwei Zeilen reichen kann}
\begin{document}
\maketitle

%�berschrift des Literaturverzeichnisses - delete this line in english
\renewcommand{\refname}{Literaturverzeichnis}
\setcounter{footnote}{2} %Auf Anzahl der AutorInnen setzen, damit die weitere Nummerierung der Fu�noten passt

\begin{abstract}
Die \LaTeX-Klasse \texttt{lni} setzt die Layout-Vorgaben f�r Beitr�ge in LNI Konferenzb�nden um. Dieses Dokument beschreibt ihre Verwendung und ist ein Beispiel f�r die entsprechende Darstellung. Der Abstract ist ein kurzer �berblick �ber die Arbeit der zwischen 70 und 150 W�rtern lang sein und das Wichtigste enthalten sollte. Die Formatierung erfolgt automatisch innerhalb des abstract-Bereichs.
\end{abstract}
\begin{keywords}
LNI Guidelines, \LaTeX Vorlage. Die Formatierung erfolgt automatisch innerhalb des keywords-Bereichs.
\end{keywords}



\section{Title}

\subsection{Subtitle}
 
Plain text.

\subsection{Another subtitle}

More plain text.
 

\bibliographystyle{style/lnig}
\bibliography{references}

\end{document}
