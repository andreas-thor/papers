\documentclass{style/lni}
% in englisch stattdessen:
%\documentclass[english]{lni}

\IfFileExists{style/latin1.sty}{\usepackage{style/latin1}}{\usepackage{isolatin1}}

\usepackage{graphicx}
\usepackage{style/fancyhdr}
\usepackage{listings} %if lstlistings is used
\usepackage{style/changepage} %for changing topmargin on first page
\usepackage[figurename=Abb., tablename=Tab., small]{caption}[2008/04/01]
\renewcommand{\lstlistingname}{List.}    % Listingname hei�t nun List. 
 
%Beginn der Seitenz�hlung f�r diesen Beitrag
\setcounter{page}{11} 
  
%Kopfzeileneinstellungen
\pagestyle{fancy}
\fancyhead{} % L�scht alle Kopfzeileneinstellungen
\fancyhead[RO]{\small $<$Vorname Nachname [et. al.]$>$(Hrsg.): $<$Buchtitel$>$, \linebreak Lecture Notes in Informatics (LNI), Gesellschaft f�r Informatik, Bonn $<$Jahr$>$ \hspace{5pt} \thepage \hspace{0.05cm}}
\fancyfoot{} % L�scht alle Fu�zeileneinstellungen
\renewcommand{\headrulewidth}{0.4pt} %Linie unter Kopfzeile  
\setcounter{footnote}{0}

\author{
Luise Kaufmann\footnote{Telekom Security, luise.kaufmann@telekom.de} \, 
Tobias Welz\footnote{Hochschule f�r Telekommunikation Leipzig, welz@hft-leipzig.de} \,
Andreas Thor\footnote{Hochschule f�r Telekommunikation Leipzig, thor@hft-leipzig.de}}
\title{DIAL -- Ein BigBlueButton-basiertes System f�r dezentrale, interaktive
Vorlesungen}
\begin{document}
\maketitle

%�berschrift des Literaturverzeichnisses - delete this line in english
\renewcommand{\refname}{Literaturverzeichnis}
\setcounter{footnote}{2} %Auf Anzahl der AutorInnen setzen, damit die weitere Nummerierung der Fu�noten passt

\begin{abstract}
Die \LaTeX-Klasse \texttt{lni} setzt die Layout-Vorgaben f�r Beitr�ge in LNI Konferenzb�nden um. Dieses Dokument beschreibt ihre Verwendung und ist ein Beispiel f�r die entsprechende Darstellung. Der Abstract ist ein kurzer �berblick �ber die Arbeit der zwischen 70 und 150 W�rtern lang sein und das Wichtigste enthalten sollte. Die Formatierung erfolgt automatisch innerhalb des abstract-Bereichs.
\end{abstract}
\begin{keywords}
E-Learning, BigBlueButton
\end{keywords}



\section{Motivation}


{\bf DIAL} = \underline{D}istributed \underline{I}nter\underline{A}ctive \underline{L}ecture

Szenarien

a) Zu kleine R�ume in Hochschule / zu gro�e Gruppen
Wachsende Studierendenzahlen REFERENZ
Bauliche / r�umliche Beschr�nkungen
Aufteilen von Studiengruppen nicht immer m�glich, z.B. wegen Begrenzun des Lehrdeputats

b) Hochschulen mit mehreren Standorten 
Finanzielle Zw�nge f�rdern Zusammenlegungen bzw. Kooperationen von Hochschulen bzw. Standorten um Synergieeffekte zu nutzen
Vorlesung nur von einem Standort aus
  
c) Verteilte Lerngruppen (im Gegensatz zu Einzelteilnehmern)
Beispiel Ausbildungszentren der Deutschen Telekom, bei denen die deutschlandweit verteilten Studierenden in regionale Gruppen
Kombination lokalen Lerngrupper mit E-Learning;
F�r alle ``allgemeine'' Vorlesung; individuelles Lernen / �ben dann in Gruppen, u.a. gesteuert durch Lerncoach o.�.


Grunds�tzlich Szenarien auch mit Web-Konferenz wie Adobe Connect m�glich, aber Herausforderungen 

a) hohe Serveranforderungen bei vielen Leuten

b) hohe Netzwerkanforderungen wenn viele Leute im gleichen Raum / gleichen Standort 
Alle im gleichen WLAN bzw. der gleichen Funkzelle; geteilte Bandbreite

c) hohe Anforderungen an Teilnehmer (eigenes Notebook, ggf. ohne Steckdose im Dauerbetrieb)



Senken den Anforderungen durch

a) ``virtuelle Gruppennutzer''

b) nur �bertragen des Chats und der Abstimmungen

c) nur Smartphone / Tablet n�tig wenn Interaktion gew�nscht (dazwischen Stand-By)



\section{Related Work?}

Welche anderen Konzepte / Systeme f�r dezentrale Vorlesungen gibt es?


Kurzvorstellung BigBlueButton


\section{Architektur}

Setting

a) Moderator mit Notebook + Video/Mikro

b) Gastgruppen mit Notebook angemeldet; �bertragung per Beamer + Sound

c) Kommunikation der Teilnehmer via ``virtuellen Gruppennutzers'' durch Manipulation des Redis Channels



\section{Prototypische Implementation}

Was gibt es f�r Besonderheiten bei der Implementation?


\section{Use Case und Evaluation}

Vergleich der Lasttest

a) Senkung der Bandbreite im Server

b) Senkung der CPU Load beim Server; nicht so dramatisch, aber Max-Spitzen gehen weg

c) Senkung der Netzwerklast der Clients (im Raum) -- haben wir das gemessen???



Ergebnisse der Use Cases

a) 


\section{Zusammenfassung und Ausblick}

 

\bibliographystyle{style/lnig}
\bibliography{references}

\end{document}
